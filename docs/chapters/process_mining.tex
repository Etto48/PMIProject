\section{Process mining}
\label{sec:process_mining}

We mined the logs generated by the simulation of the collapsed workflow.

We modified the simulation configuration to make the 100 tokens
flow through every path of the workflow. The most important gateways that we
changed are listed in the following table.
\begin{table}[H]
\centering
\begin{tabular}{|l|r|r|}
\hline
\textbf{Gateway} & \textbf{Yes} & \textbf{No} \\
\hline
RAW SESSION INVALID & 5\% & 95\% \\
\hline
RECORD SUFFICIENT & 95\% & 5\% \\
\hline
SESSION SUFFICIENT & 95\% & 5\% \\
\hline
IS FIRST SESSION & 20\% & 80\% \\
\hline
COVERAGE SATISFYING & 70\% & 30\% \\
\hline
DEVELOPMENT PHASE & 70\% & 30\% \\
\hline
\end{tabular}
\caption{Gateways configuration}
\label{tab:gateways_configuration}
\end{table}

\begin{figure}[H]
\centering
\includegraphics[width=0.8\textwidth]{figures/disco_map.pdf}
\caption{Disco analysis}
\label{fig:disco_analysis}
\end{figure}

\begin{figure}[H]
\centering
\includegraphics[width=\textwidth]{figures/apromore_map.pdf}
\caption{Apromore analysis}
\label{fig:apromore_analysis}
\end{figure}

As we can see, the two transition maps mined from Disco and from 
Apromore are identical. The only difference stays in the frequencies
because in Disco the frequencies are calculated as the total number of
times a transition is executed, even on the same token; while in Apromore
the frequencies are calculated as the number of individual tokens that execute a
transition. This behavior can be changed with a setting in both tools.

\begin{figure}[H]
\centering
\includegraphics[width=\textwidth]{figures/prom_mined.pdf}
\caption{ProM mined BPMN model}
\label{fig:prom_mined}
\end{figure}

We mined the logs using the "Heuristics Miner ProM6" mining algorithm.

\begin{figure}[H]
\centering
\includegraphics[width=\textwidth]{figures/apromore_mined.pdf}
\caption{Apromore mined BPMN model}
\label{fig:apromore_mined}
\end{figure}

The BPMN model mined from Apromore is more detailed and
covers more cases than the one mined from ProM.
The key differences between the ProM model and the Apromore one are that the
ProM model is missing the paths that skip
the training and the configuration as well as one of the two paths that skip only
the training. Furthermore, the training loop is much simpler in the ProM model, as
it is missing every path that restarts the training after
"CHECK VALIDATION REPORT".

\begin{table}[H]
\centering
\begin{tabular}{|r|l|l|l|l|}
\hline
\textbf{Tool} & \textbf{Trace} & \textbf{Generalization} & \textbf{Precision} & \textbf{Simplicity} \\
\hline
Apromore & 0.4203 & 0.9872 & 0.7566 & 62 \\
\hline
ProM & 0.2917 & 0.9871 & 0.9871 & 39 \\
\hline
\end{tabular}
\caption{Comparison of the process mining tools}
\label{tab:process_mining_comparison}
\end{table}

\subsection{Violations}
\label{sec:mining_violations}

We modified the logs to introduce 3 violations in the workflow. The
violations are the following:
\begin{itemize}
    \item Skipping the dataset creation ("CHECK DATA BALANCING"
        and "CHECK INPUT COVERAGE") using data from another user.
    \item Skipping "SET \# ITERATIONS" and "CHECK LEARNING PLOT" because we are
        using early stopping.
    \item Skipping "CHECK DATA BALANCING" because we are using a resampling
        technique.
\end{itemize}

Each violation is introduced 3 times in the logs.